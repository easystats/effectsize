%  LaTeX support: latex@mdpi.com
%  In case you need support, please attach all files that are necessary for compiling as well as the log file, and specify the details of your LaTeX setup (which operating system and LaTeX version / tools you are using).

%=================================================================
\documentclass[mathematics,article,submit,moreauthors,pdftex]{mdpi}

% If you would like to post an early version of this manuscript as a preprint, you may use preprint as the journal and change 'submit' to 'accept'. The document class line would be, e.g., \documentclass[preprints,article,accept,moreauthors,pdftex]{mdpi}. This is especially recommended for submission to arXiv, where line numbers should be removed before posting. For preprints.org, the editorial staff will make this change immediately prior to posting.

%% Some pieces required from the pandoc template
\setlist[itemize]{leftmargin=*,labelsep=5.8mm}
\setlist[enumerate]{leftmargin=*,labelsep=4.9mm}


%--------------------
% Class Options:
%--------------------
%----------
% journal
%----------
% Choose between the following MDPI journals:
% acoustics, actuators, addictions, admsci, aerospace, agriculture, agriengineering, agronomy, algorithms, animals, antibiotics, antibodies, antioxidants, applsci, arts, asc, asi, atmosphere, atoms, axioms, batteries, bdcc, behavsci , beverages, bioengineering, biology, biomedicines, biomimetics, biomolecules, biosensors, brainsci , buildings, cancers, carbon , catalysts, cells, ceramics, challenges, chemengineering, chemistry, chemosensors, children, cleantechnol, climate, clockssleep, cmd, coatings, colloids, computation, computers, condensedmatter, cosmetics, cryptography, crystals, dairy, data, dentistry, designs , diagnostics, diseases, diversity, drones, econometrics, economies, education, electrochem, electronics, energies, entropy, environments, epigenomes, est, fermentation, fibers, fire, fishes, fluids, foods, forecasting, forests, fractalfract, futureinternet, futurephys, galaxies, games, gastrointestdisord, gels, genealogy, genes, geohazards, geosciences, geriatrics, hazardousmatters, healthcare, heritage, highthroughput, horticulturae, humanities, hydrology, ijerph, ijfs, ijgi, ijms, ijns, ijtpp, informatics, information, infrastructures, inorganics, insects, instruments, inventions, iot, j, jcdd, jcm, jcp, jcs, jdb, jfb, jfmk, jimaging, jintelligence, jlpea, jmmp, jmse, jnt, jof, joitmc, jpm, jrfm, jsan, land, languages, laws, life, literature, logistics, lubricants, machines, magnetochemistry, make, marinedrugs, materials, mathematics, mca, medicina, medicines, medsci, membranes, metabolites, metals, microarrays, micromachines, microorganisms, minerals, modelling, molbank, molecules, mps, mti, nanomaterials, ncrna, neuroglia, nitrogen, notspecified, nutrients, ohbm, particles, pathogens, pharmaceuticals, pharmaceutics, pharmacy, philosophies, photonics, physics, plants, plasma, polymers, polysaccharides, preprints , proceedings, processes, proteomes, psych, publications, quantumrep, quaternary, qubs, reactions, recycling, religions, remotesensing, reports, resources, risks, robotics, safety, sci, scipharm, sensors, separations, sexes, signals, sinusitis, smartcities, sna, societies, socsci, soilsystems, sports, standards, stats, surfaces, surgeries, sustainability, symmetry, systems, technologies, test, toxics, toxins, tropicalmed, universe, urbansci, vaccines, vehicles, vetsci, vibration, viruses, vision, water, wem, wevj

%---------
% article
%---------
% The default type of manuscript is "article", but can be replaced by:
% abstract, addendum, article, benchmark, book, bookreview, briefreport, casereport, changes, comment, commentary, communication, conceptpaper, conferenceproceedings, correction, conferencereport, expressionofconcern, extendedabstract, meetingreport, creative, datadescriptor, discussion, editorial, essay, erratum, hypothesis, interestingimages, letter, meetingreport, newbookreceived, obituary, opinion, projectreport, reply, retraction, review, perspective, protocol, shortnote, supfile, technicalnote, viewpoint
% supfile = supplementary materials

%----------
% submit
%----------
% The class option "submit" will be changed to "accept" by the Editorial Office when the paper is accepted. This will only make changes to the frontpage (e.g., the logo of the journal will get visible), the headings, and the copyright information. Also, line numbering will be removed. Journal info and pagination for accepted papers will also be assigned by the Editorial Office.

%------------------
% moreauthors
%------------------
% If there is only one author the class option oneauthor should be used. Otherwise use the class option moreauthors.

%---------
% pdftex
%---------
% The option pdftex is for use with pdfLaTeX. If eps figures are used, remove the option pdftex and use LaTeX and dvi2pdf.

%=================================================================
\firstpage{1}
\makeatletter
\setcounter{page}{\@firstpage}
\makeatother
\pubvolume{xx}
\issuenum{1}
\articlenumber{5}
\pubyear{2019}
\copyrightyear{2019}
%\externaleditor{Academic Editor: name}
\history{Received: date; Accepted: date; Published: date}
\updates{yes} % If there is an update available, un-comment this line

%% MDPI internal command: uncomment if new journal that already uses continuous page numbers
%\continuouspages{yes}

%------------------------------------------------------------------
% The following line should be uncommented if the LaTeX file is uploaded to arXiv.org
%\pdfoutput=1

%=================================================================
% Add packages and commands here. The following packages are loaded in our class file: fontenc, calc, indentfirst, fancyhdr, graphicx, lastpage, ifthen, lineno, float, amsmath, setspace, enumitem, mathpazo, booktabs, titlesec, etoolbox, amsthm, hyphenat, natbib, hyperref, footmisc, geometry, caption, url, mdframed, tabto, soul, multirow, microtype, tikz

%=================================================================
%% Please use the following mathematics environments: Theorem, Lemma, Corollary, Proposition, Characterization, Property, Problem, Example, ExamplesandDefinitions, Hypothesis, Remark, Definition
%% For proofs, please use the proof environment (the amsthm package is loaded by the MDPI class).

%=================================================================
% Full title of the paper (Capitalized)
\Title{Phi, Fei, Fo, Fum: Effect Sizes for Chi-squared Tests}

% Authors, for the paper (add full first names)
\Author{Mattan S. Ben-Shachar$^{1,
*}$\href{https://orcid.org/0000-0002-4287-4801}{\orcidicon}, Indrajeet
Patil$^{2}$\href{https://orcid.org/0000-0003-1995-6531}{\orcidicon}, Rémi
Thériault$^{3}$\href{https://orcid.org/0000-0003-4315-6788}{\orcidicon}, Brenton
M.
Wiernik$^{4}$\href{https://orcid.org/0000-0001-9560-6336}{\orcidicon}, Daniel
Lüdecke$^{5}$\href{https://orcid.org/0000-0002-8895-3206}{\orcidicon}}

% Authors, for metadata in PDF
\AuthorNames{Mattan S. Ben-Shachar, Indrajeet Patil, Rémi
Thériault, Brenton M. Wiernik, Daniel Lüdecke}

% Affiliations / Addresses (Add [1] after \address if there is only one affiliation.)
\address{%
$^{1}$ \quad Independent researcher; \\
$^{2}$ \quad Center for Humans and Machines, Max Planck Institute for
Human Development, Berlin, Germany; \\
$^{3}$ \quad Department of Psychology, Université du Québec à Montréal,
Montréal, Québec, Canada; \\
$^{4}$ \quad Independent researcher; \\
$^{5}$ \quad Institute of Medical Sociology, University Medical Center
Hamburg-Eppendorf, Germany; \\
}
% Contact information of the corresponding author
\corres{Correspondence: mattan-mail}

% Current address and/or shared authorship








% The commands \thirdnote{} till \eighthnote{} are available for further notes

% Simple summary
\simplesumm{The \emph{\{effectsize\}} package from the \emph{easystats}
ecosystem makes it easy to estimate effect sizes in R for common
\(\chi^2\) tests, including a new effect size, Fei, for goodness-of-fit
tests.}

% Abstract (Do not insert blank lines, i.e. \\)
\abstract{In both theoretical and applied research, it is often of
interest to assess the strength of an observed association. Existing
guidelines also frequently recommend going beyond null-hypothesis
significance testing and to report effect sizes and their confidence
intervals. As such, measures of effect sizes are increasingly reported,
valued, and understood. Beyond their value in shaping the interpretation
of the results from a given study, reporting effect sizes is critical
for meta-analyses, which rely on their aggregation. We here review the
most common effect sizes for analyses of categorical variables that use
the \(\chi^2\) (chi-square) statistic, and introduce a new effect
size---פ (Fei, pronounced /fej/ or ``fay''). We demonstrate the
implementation of these measures and their confidence intervals via the
\texttt{\{effectsize\}} package \citep{benshachar2020effectsize} in the
R programming language.}

% Keywords
\keyword{R; easystats}

% The fields PACS, MSC, and JEL may be left empty or commented out if not applicable
%\PACS{J0101}
%\MSC{}
%\JEL{}

%%%%%%%%%%%%%%%%%%%%%%%%%%%%%%%%%%%%%%%%%%
% Only for the journal Diversity
%\LSID{\url{http://}}

%%%%%%%%%%%%%%%%%%%%%%%%%%%%%%%%%%%%%%%%%%
% Only for the journal Applied Sciences:
%\featuredapplication{Authors are encouraged to provide a concise description of the specific application or a potential application of the work. This section is not mandatory.}
%%%%%%%%%%%%%%%%%%%%%%%%%%%%%%%%%%%%%%%%%%

%%%%%%%%%%%%%%%%%%%%%%%%%%%%%%%%%%%%%%%%%%
% Only for the journal Data:
%\dataset{DOI number or link to the deposited data set in cases where the data set is published or set to be published separately. If the data set is submitted and will be published as a supplement to this paper in the journal Data, this field will be filled by the editors of the journal. In this case, please make sure to submit the data set as a supplement when entering your manuscript into our manuscript editorial system.}

%\datasetlicense{license under which the data set is made available (CC0, CC-BY, CC-BY-SA, CC-BY-NC, etc.)}

%%%%%%%%%%%%%%%%%%%%%%%%%%%%%%%%%%%%%%%%%%
% Only for the journal Toxins
%\keycontribution{The breakthroughs or highlights of the manuscript. Authors can write one or two sentences to describe the most important part of the paper.}

%\setcounter{secnumdepth}{4}
%%%%%%%%%%%%%%%%%%%%%%%%%%%%%%%%%%%%%%%%%%

% Pandoc syntax highlighting
\usepackage{color}
\usepackage{fancyvrb}
\newcommand{\VerbBar}{|}
\newcommand{\VERB}{\Verb[commandchars=\\\{\}]}
\DefineVerbatimEnvironment{Highlighting}{Verbatim}{commandchars=\\\{\}}
% Add ',fontsize=\small' for more characters per line
\usepackage{framed}
\definecolor{shadecolor}{RGB}{248,248,248}
\newenvironment{Shaded}{\begin{snugshade}}{\end{snugshade}}
\newcommand{\AlertTok}[1]{\textcolor[rgb]{0.94,0.16,0.16}{#1}}
\newcommand{\AnnotationTok}[1]{\textcolor[rgb]{0.56,0.35,0.01}{\textbf{\textit{#1}}}}
\newcommand{\AttributeTok}[1]{\textcolor[rgb]{0.77,0.63,0.00}{#1}}
\newcommand{\BaseNTok}[1]{\textcolor[rgb]{0.00,0.00,0.81}{#1}}
\newcommand{\BuiltInTok}[1]{#1}
\newcommand{\CharTok}[1]{\textcolor[rgb]{0.31,0.60,0.02}{#1}}
\newcommand{\CommentTok}[1]{\textcolor[rgb]{0.56,0.35,0.01}{\textit{#1}}}
\newcommand{\CommentVarTok}[1]{\textcolor[rgb]{0.56,0.35,0.01}{\textbf{\textit{#1}}}}
\newcommand{\ConstantTok}[1]{\textcolor[rgb]{0.00,0.00,0.00}{#1}}
\newcommand{\ControlFlowTok}[1]{\textcolor[rgb]{0.13,0.29,0.53}{\textbf{#1}}}
\newcommand{\DataTypeTok}[1]{\textcolor[rgb]{0.13,0.29,0.53}{#1}}
\newcommand{\DecValTok}[1]{\textcolor[rgb]{0.00,0.00,0.81}{#1}}
\newcommand{\DocumentationTok}[1]{\textcolor[rgb]{0.56,0.35,0.01}{\textbf{\textit{#1}}}}
\newcommand{\ErrorTok}[1]{\textcolor[rgb]{0.64,0.00,0.00}{\textbf{#1}}}
\newcommand{\ExtensionTok}[1]{#1}
\newcommand{\FloatTok}[1]{\textcolor[rgb]{0.00,0.00,0.81}{#1}}
\newcommand{\FunctionTok}[1]{\textcolor[rgb]{0.00,0.00,0.00}{#1}}
\newcommand{\ImportTok}[1]{#1}
\newcommand{\InformationTok}[1]{\textcolor[rgb]{0.56,0.35,0.01}{\textbf{\textit{#1}}}}
\newcommand{\KeywordTok}[1]{\textcolor[rgb]{0.13,0.29,0.53}{\textbf{#1}}}
\newcommand{\NormalTok}[1]{#1}
\newcommand{\OperatorTok}[1]{\textcolor[rgb]{0.81,0.36,0.00}{\textbf{#1}}}
\newcommand{\OtherTok}[1]{\textcolor[rgb]{0.56,0.35,0.01}{#1}}
\newcommand{\PreprocessorTok}[1]{\textcolor[rgb]{0.56,0.35,0.01}{\textit{#1}}}
\newcommand{\RegionMarkerTok}[1]{#1}
\newcommand{\SpecialCharTok}[1]{\textcolor[rgb]{0.00,0.00,0.00}{#1}}
\newcommand{\SpecialStringTok}[1]{\textcolor[rgb]{0.31,0.60,0.02}{#1}}
\newcommand{\StringTok}[1]{\textcolor[rgb]{0.31,0.60,0.02}{#1}}
\newcommand{\VariableTok}[1]{\textcolor[rgb]{0.00,0.00,0.00}{#1}}
\newcommand{\VerbatimStringTok}[1]{\textcolor[rgb]{0.31,0.60,0.02}{#1}}
\newcommand{\WarningTok}[1]{\textcolor[rgb]{0.56,0.35,0.01}{\textbf{\textit{#1}}}}

% tightlist command for lists without linebreak
\providecommand{\tightlist}{%
  \setlength{\itemsep}{0pt}\setlength{\parskip}{0pt}}




\begin{document}


%%%%%%%%%%%%%%%%%%%%%%%%%%%%%%%%%%%%%%%%%%

\hypertarget{introduction}{%
\section{Introduction}\label{introduction}}

Over the last two decades, there has been growing concerns about the
so-called replication crisis in psychology and other fields
\citep{OSC2015estimating, camerer2018evaluating}. As a result, the
scientific community has paid increasing attention to the issue of
replicability in science, as well as to to good research and statistical
practices.

In this context, many have highlighted the limitations of
null-hypothesis significance testing and called for more modern
approaches to statistics \citep{cumming2014new}. One such recommendation
coming for example from the ``New Statistics'' movement is to report
effect sizes and their corresponding confidence intervals, and to
increasingly rely on meta-analyses to increase confidence in those
estimations. These recommendations are meant to complement (or even
replace, according to some) null-hypothesis significance testing and
would help transition toward a ``cumulative quantitative discipline''.

These so-called ``New Statistics'' are synergistic because effect sizes
are not only useful for interpreting study results in themselves, but
also because they are necessary for meta-analyses, which aggregate
effect sizes and their confidence intervals to create a summary effect
size of its own \citep[\citet{wiernik2020unbiased}]{degeest2010impact}.

Unfortunately, popular software do not always offer the necessary
implementations of the specialized effect sizes necessary for a given
research design and their confidence intervals. In this paper, we review
the most commonly used effect sizes for analyses of categorical
variables that use the \(\chi^2\) (chi-square) test statistic, and
introduce a new effect size---פ (Fei, pronounced /fej/ or ``fay'').

Importantly, we offer researchers an applied walkthrough on how to use
these effect sizes in practice thanks to the \texttt{\{effectsize\}}
package \citep{benshachar2020effectsize} in the R programming language
\citep{base2023}, which implements these measures and their confidence
intervals. We cover in turn tests of independence (\emph{φ}/phi,
Cramér's \emph{V}) and tests of goodness of fit (Cohen's \emph{w} and a
new proposed effect size, פ/Fei).

\hypertarget{tests-of-independence}{%
\section{Tests of Independence}\label{tests-of-independence}}

The \(\chi^2\) test of independence between two categorical variables
examines if the frequency distribution of one of the variables is
dependent on the other. That is, are the two variables correlated such
that, for example, members of group 1 on variable X are more likely to
be members of group A on variable Y, rather than evenly spread across Y
variable groups A and B. Formally, the test examines how likely the
observed conditional frequencies (cell frequencies) are under the null
hypotheses of independence. This is done by examining the degree the
observed cell frequencies deviate from the frequencies that would be
expected if the variables were indeed independent. The test statistic
for these tests is the \(\chi^2\), which is computed as:

\[
\chi^2 = \sum_{i=1}^{l\times k}{\frac{(O_i-E_i)^2}{E_i}}
\]

Where \(O_i\) are the \emph{observed} frequencies and \(E_i\) are the
frequencies \emph{expected} under independence, and \(l\) and \(k\) are
the number of rows and columns of the contingency table.

Instead of the deviations between the observed and expected frequencies,
we can write \(\chi^2\) in terms of observed and expected cell
\emph{probabilities} and the total sample size \(N\) (since \(p=k/N\)):

\[
\chi^2 = N\times\sum_{i=1}^{l\times k}{\frac{(p_{O_i}-p_{E_i})^2}{p_{E_i}}}
\]

Where \(p_{O_i}\) are the \emph{observed} cell probabilities and
\(p_{E_i}\) are the probabilities \emph{expected} under independence.

Here is a short example in R to demonstrate whether the probability of
survival of the sinking of the Titanic is dependant on the sex of the
passenger. The null hypothesis tested here is that the probability of
survival is independent of the passenger's sex.

\begin{Shaded}
\begin{Highlighting}[]
\NormalTok{(Titanic\_xtab }\OtherTok{\textless{}{-}} \FunctionTok{as.table}\NormalTok{(}\FunctionTok{apply}\NormalTok{(Titanic, }\FunctionTok{c}\NormalTok{(}\DecValTok{2}\NormalTok{, }\DecValTok{4}\NormalTok{), sum)))}
\end{Highlighting}
\end{Shaded}

\begin{verbatim}
        Survived
Sex        No  Yes
  Male   1364  367
  Female  126  344
\end{verbatim}

\begin{Shaded}
\begin{Highlighting}[]
\FunctionTok{chisq.test}\NormalTok{(Titanic\_xtab)}
\end{Highlighting}
\end{Shaded}

\begin{verbatim}

    Pearson's Chi-squared test with Yates' continuity correction

data:  Titanic_xtab
X-squared = 454.5, df = 1, p-value < 2.2e-16
\end{verbatim}

The performed \(\chi^2\)-test is statistically significant, thus we can
reject the hypothesis of independence. However, the output includes no
effect size. We cannot draw conclusions of the strength of the
association between sex and survival.

\hypertarget{phi}{%
\subsection{Phi}\label{phi}}

For a 2-by-2 contingency table analysis, like the one used above, the
\(\phi\) (\emph{phi}) coefficient is a correlation-like measure of
effect size indicating the strength of association between the two
binary variables. One way to compute this effect size is to re-code the
binary variables as dummy (0, 1) variables, and computing the Pearson
correlation between them:

\[
\phi = |r_{AB}|
\]

Another way to compute \(\phi\) is by using the \(\chi^2\) statistic:

\[
\phi = \sqrt{\frac{\chi^2}{N}} = \sqrt{\sum_{i=1}^{l\times k}{\frac{(p_{O_i}-p_{E_i})^2}{p_{E_i}}}}
\]

This value ranges between 0 (no association) and 1 (complete
dependence), and its values can be interpreted the same as Person's
correlation coefficient.

\begin{Shaded}
\begin{Highlighting}[]
\FunctionTok{library}\NormalTok{(effectsize)}
\FunctionTok{library}\NormalTok{(correlation)}

\FunctionTok{phi}\NormalTok{(Titanic\_xtab, }\AttributeTok{adjust =} \ConstantTok{FALSE}\NormalTok{)}
\end{Highlighting}
\end{Shaded}

\begin{verbatim}
Phi  |       95% CI
-------------------
0.46 | [0.42, 1.00]

- One-sided CIs: upper bound fixed at [1.00].
\end{verbatim}

\begin{Shaded}
\begin{Highlighting}[]
\NormalTok{tidyr}\SpecialCharTok{::}\FunctionTok{uncount}\NormalTok{(}\FunctionTok{as.data.frame}\NormalTok{(Titanic\_xtab), }\AttributeTok{weights =}\NormalTok{ Freq) }\SpecialCharTok{|\textgreater{}}
  \FunctionTok{transform}\NormalTok{(}\AttributeTok{Survived =}\NormalTok{ Survived }\SpecialCharTok{==} \StringTok{"Yes"}\NormalTok{, }\AttributeTok{Sex =}\NormalTok{ Sex }\SpecialCharTok{==} \StringTok{"Male"}\NormalTok{) }\SpecialCharTok{|\textgreater{}}
  \FunctionTok{correlation}\NormalTok{()}
\end{Highlighting}
\end{Shaded}

\begin{verbatim}
# Correlation Matrix (pearson-method)

Parameter1 | Parameter2 |     r |         95% CI | t(2199) |         p
----------------------------------------------------------------------
Sex        |   Survived | -0.46 | [-0.49, -0.42] |  -24.00 | < .001***

p-value adjustment method: Holm (1979)
Observations: 2201
\end{verbatim}

Note that \(\phi\) cannot be negative, so will take the \emph{absolute}
value of Pearson's correlation coefficient. Also note that
\texttt{\{effectsize\}} gives a \emph{one-sided} confidence interval by
default, to match the positive direction of the associated \(\chi^2\)
test at \(\alpha = 0.05\) (that the association is \emph{larger} than 0
at a 95\% confidence level).

\hypertarget{cramuxe9rs-v-and-tschuprows-t}{%
\subsection{\texorpdfstring{Cramér's \emph{V} (and Tschuprow's
\emph{T})}{Cramér's V (and Tschuprow's T)}}\label{cramuxe9rs-v-and-tschuprows-t}}

When the contingency table is larger than 2-by-2, using
\(\sqrt{\chi^2/N}\) can produce values larger than 1, and so loses its
interpretability as a correlation like effect size. Cramér showed
\citep{cramer1999mathematical} that while for 2-by-2 the maximal
possible value of \(\chi^2\) is \(N\), for larger tables the maximal
possible value for \(\chi^2\) is \(N\times (\text{min}(k,l)-1)\).
Therefore, he suggested the \(V\) effect size (also sometimes known as
Cramér's phi and denoted as \(\phi_{c}\)):

\[
\text{Cramer's } V = \sqrt{\frac{\chi^2}{N(\text{min}(k,l)-1)}}
\]

\(V\) is 1 when the columns are completely dependent on the rows, or the
rows are completely dependent on the columns (and 0 when rows and
columns are completely independent).

\begin{Shaded}
\begin{Highlighting}[]
\NormalTok{(Titanic\_xtab2 }\OtherTok{\textless{}{-}} \FunctionTok{as.table}\NormalTok{(}\FunctionTok{apply}\NormalTok{(Titanic, }\FunctionTok{c}\NormalTok{(}\DecValTok{1}\NormalTok{, }\DecValTok{4}\NormalTok{), sum)))}
\end{Highlighting}
\end{Shaded}

\begin{verbatim}
      Survived
Class   No Yes
  1st  122 203
  2nd  167 118
  3rd  528 178
  Crew 673 212
\end{verbatim}

\begin{Shaded}
\begin{Highlighting}[]
\FunctionTok{cramers\_v}\NormalTok{(Titanic\_xtab2, }\AttributeTok{adjust =} \ConstantTok{FALSE}\NormalTok{)}
\end{Highlighting}
\end{Shaded}

\begin{verbatim}
Cramer's V |       95% CI
-------------------------
0.29       | [0.26, 1.00]

- One-sided CIs: upper bound fixed at [1.00].
\end{verbatim}

Tschuprow \citep{tschuprow1939principles} devised an alternative value,
at

\[
\text{Tschuprow's } T = \sqrt{\frac{\chi^2}{N\sqrt{(k-1)(l-1)}}}
\]

which is 1 only when the columns are completely dependent on the rows
\emph{and} the rows are completely dependent on the columns, which is
only possible when the contingency table is a square.

For example, in the following table, each row is dependent on the column
value; that is, if we know if the food is a soy, milk or meat product,
we also know if the food is vegan or not. However, the columns are
\emph{not} fully dependent on the rows: knowing the food is vegan tells
us the food is soy based, however knowing it is not vegan does not allow
us to classify the food - it can be either a milk product or a meat
product.

\begin{Shaded}
\begin{Highlighting}[]
\FunctionTok{data}\NormalTok{(}\StringTok{"food\_class"}\NormalTok{)}
\NormalTok{food\_class}
\end{Highlighting}
\end{Shaded}

\begin{verbatim}
          Soy Milk Meat
Vegan      47    0    0
Not-Vegan   0   12   21
\end{verbatim}

Accordingly, in such a table, Cramer's \emph{V} will be 1, but
Tschuprow's \emph{T} will not be:

\begin{Shaded}
\begin{Highlighting}[]
\FunctionTok{cramers\_v}\NormalTok{(food\_class, }\AttributeTok{adjust =} \ConstantTok{FALSE}\NormalTok{)}
\end{Highlighting}
\end{Shaded}

\begin{verbatim}
Cramer's V |       95% CI
-------------------------
1.00       | [0.81, 1.00]

- One-sided CIs: upper bound fixed at [1.00].
\end{verbatim}

\begin{Shaded}
\begin{Highlighting}[]
\FunctionTok{tschuprows\_t}\NormalTok{(food\_class)}
\end{Highlighting}
\end{Shaded}

\begin{verbatim}
Tschuprow's T |       95% CI
----------------------------
0.84          | [0.68, 1.00]

- One-sided CIs: upper bound fixed at [1.00].
\end{verbatim}

We can generalize \(\phi\), \(V\), and \(T\) to:
\(\sqrt{\frac{\chi^2}{\chi^2_{\text{max}}}}\). That is, they are express
a proportional of the sample-\(\chi^2\) to the maximally possible
\(\chi^2\) given the study design.

These coefficients can also be used for confusion matrices - 2-by-2
contingency tables used in assessing machine learning algorithms
classification abilities, comparing true outcome classes with the
model-predicted outcome class. In fact, a popular metric is the Matthews
correlation coefficient (MCC) for binary classifiers, which is often
presented in terms of true and false positives and negatives, is nothing
more that \(\phi\) \citep{chicco2020advantages}.

\hypertarget{goodness-of-fit}{%
\section{Goodness of Fit}\label{goodness-of-fit}}

These tests compare an observed distribution of a multinomial variable
to an expected distribution, using the same \(\chi^2\) statistic. Here
too we can compute an effect size as
\(\sqrt{\frac{\chi^2}{\chi^2_{\text{max}}}}\), all we need to find is
\(\chi^2_{\text{max}}\).

\hypertarget{cohens-w}{%
\subsection{\texorpdfstring{Cohen's
\emph{w}}{Cohen's w}}\label{cohens-w}}

Cohen \citep{cohen2013statistical} defined an effect
size---\emph{w}---for the goodness of fit test:

\[
\text{Cohen's } w = \sqrt{\sum_{i=1}^{k}{\frac{(p_{O_i}-p_{E_i})^2}{p_{E_i}}}} = \sqrt{\frac{\chi^2}{N}}
\]

Thus, \(\chi^2_\text{max} = N\).

\begin{Shaded}
\begin{Highlighting}[]
\NormalTok{(Titanic\_freq }\OtherTok{\textless{}{-}} \FunctionTok{as.table}\NormalTok{(}\FunctionTok{apply}\NormalTok{(Titanic, }\DecValTok{2}\NormalTok{, sum)))}
\end{Highlighting}
\end{Shaded}

\begin{verbatim}
  Male Female 
  1731    470 
\end{verbatim}

\begin{Shaded}
\begin{Highlighting}[]
\NormalTok{p\_E }\OtherTok{\textless{}{-}} \FunctionTok{c}\NormalTok{(}\FloatTok{0.5}\NormalTok{, }\FloatTok{0.5}\NormalTok{)}

\FunctionTok{cohens\_w}\NormalTok{(Titanic\_freq, }\AttributeTok{p =}\NormalTok{ p\_E)}
\end{Highlighting}
\end{Shaded}

\begin{verbatim}
Cohen's w |       95% CI
------------------------
0.57      | [0.54, 1.00]

- One-sided CIs: upper bound fixed at [1.00].
\end{verbatim}

Unfortunately, \emph{w} has an upper bound of 1 \emph{only} when the
variable is binomial (has two categories) and the expected distribution
is uniform (\(p = 1 - p = 0.5\)). If the distribution is none uniform
\citep{rosenberg2010generalized} or if there are more than 2 classes
\citep{johnston2006measures}, then \(\chi^2_\text{max} > N\), and so
\emph{w} can be larger than 1.

\begin{Shaded}
\begin{Highlighting}[]
\NormalTok{O }\OtherTok{\textless{}{-}} \FunctionTok{c}\NormalTok{(}\DecValTok{90}\NormalTok{, }\DecValTok{10}\NormalTok{)}
\NormalTok{p\_E }\OtherTok{\textless{}{-}} \FunctionTok{c}\NormalTok{(}\FloatTok{0.35}\NormalTok{, }\FloatTok{0.65}\NormalTok{)}
\FunctionTok{cohens\_w}\NormalTok{(O, }\AttributeTok{p =}\NormalTok{ p\_E)}
\end{Highlighting}
\end{Shaded}

\begin{verbatim}
Cohen's w |       95% CI
------------------------
1.15      | [0.99, 1.36]

- One-sided CIs: upper bound fixed at [1.36~].
\end{verbatim}

\begin{Shaded}
\begin{Highlighting}[]
\NormalTok{O }\OtherTok{\textless{}{-}} \FunctionTok{c}\NormalTok{(}\DecValTok{10}\NormalTok{, }\DecValTok{20}\NormalTok{, }\DecValTok{80}\NormalTok{, }\DecValTok{5}\NormalTok{)}
\NormalTok{p\_E }\OtherTok{\textless{}{-}} \FunctionTok{c}\NormalTok{(.}\DecValTok{25}\NormalTok{, .}\DecValTok{25}\NormalTok{, .}\DecValTok{25}\NormalTok{, .}\DecValTok{25}\NormalTok{)}
\FunctionTok{cohens\_w}\NormalTok{(O, }\AttributeTok{p =}\NormalTok{ p\_E)}
\end{Highlighting}
\end{Shaded}

\begin{verbatim}
Cohen's w |       95% CI
------------------------
1.05      | [0.88, 1.73]

- One-sided CIs: upper bound fixed at [1.73~].
\end{verbatim}

Although Cohen \citep{cohen2013statistical} suggested that \emph{w} can
also be used for such designs, we believe that this hinders the
interpretation of \emph{w} since it can be arbitrarily large.

\hypertarget{fei}{%
\subsection{Fei}\label{fei}}

We present here a new effect size, פ (Fei, pronounced /fej/ or ``fay''),
which normalizes goodness-of-fit \(\chi^2\) by the proper
\(\chi^2_\text{max}\) for non-uniform and/or multinomial variables.

The largest deviation from the expected probability distribution would
occur when all observations are in the cell with the smallest expected
probability. That is:

\[
p_{O} = 
\begin{cases}
1, & \text{if } p_i = \text{min}(p) \\
0, & \text{Otherwise}
\end{cases}
\]

We can find \(\frac{(E_i-O_i)^2}{E_i}\) for each of these values:

\[
\frac{(p_{E}-p_{O})^2}{p_{E}} = 
\begin{cases}
\frac{(p_i-1)^2}{p_i} = \frac{(1-p_i)^2}{p_i}, & \text{if } p_{E} = \text{min}(p_{E}) \\
\frac{(p_i-0)^2}{p_i} = p_i, & \text{Otherwise}
\end{cases}
\]

Therefore,

\[
\begin{split}
\sum_{i=1}^{k}{\frac{(p_{O_i}-p_{E_i})^2}{p_{E_i}}} & = \sum_{i=1}^{k}{p_{E_i}} - \text{min}(p_{E}) + \frac{(1-\text{min}(p_{E}))^2}{\text{min}(p_{E})} \\
& = 1 - \text{min}(p_E) + \frac{(1-\text{min}(p_E))^2}{\text{min}(p_E)} \\
& = \frac{1-\text{min}(p_E)}{\text{min}(p_E)} \\
& = \frac{1}{\text{min}(p_E)} - 1
\end{split}
\]

And so,

\[
\begin{split}
\chi^2_\text{max} & = N \times \sum_{i=1}^{k}{\frac{(p_{O_i}-p_{E_i})^2}{p_{E_i}}} \\
 & = N \times (\frac{1}{\text{min}(p_E)} - 1)
\end{split}
\] Finally, an effect size can be derived as:

\[
\sqrt{\frac{\chi^2}{N \times (\frac{1}{\text{min}(p_E)} - 1)}}
\]

We call this effect size פ (Fei), which represents the voiceless
bilabial fricative in the Hebrew language, keeping in line with \(\phi\)
(which in modern Greek marks the same sound) and \(V\) (which in English
marks a voiced bilabial fricative; \(W\) being derived from the letter V
in modern Latin alphabet). פ will be 0 when the observed distribution
matches the expected one (under the null hypothesis) perfectly, and will
be 1 when the sample contains \emph{only} one class of
observations---the one with the smallest expected probability (under the
null hypothesis). That is, פ only when we observe only the least
expected class.

\begin{Shaded}
\begin{Highlighting}[]
\NormalTok{O }\OtherTok{\textless{}{-}} \FunctionTok{c}\NormalTok{(}\DecValTok{90}\NormalTok{, }\DecValTok{10}\NormalTok{)}
\NormalTok{p\_E }\OtherTok{\textless{}{-}} \FunctionTok{c}\NormalTok{(}\FloatTok{0.35}\NormalTok{, }\FloatTok{0.65}\NormalTok{)}
\FunctionTok{fei}\NormalTok{(O, }\AttributeTok{p =}\NormalTok{ p\_E)}
\end{Highlighting}
\end{Shaded}

\begin{verbatim}
Fei  |       95% CI
-------------------
0.85 | [0.73, 1.00]

- Adjusted for uniform expected probabilities.
- One-sided CIs: upper bound fixed at [1.00].
\end{verbatim}

\begin{Shaded}
\begin{Highlighting}[]
\NormalTok{O }\OtherTok{\textless{}{-}} \FunctionTok{c}\NormalTok{(}\DecValTok{10}\NormalTok{, }\DecValTok{20}\NormalTok{, }\DecValTok{80}\NormalTok{, }\DecValTok{5}\NormalTok{)}
\NormalTok{p\_E }\OtherTok{\textless{}{-}} \FunctionTok{c}\NormalTok{(.}\DecValTok{25}\NormalTok{, .}\DecValTok{25}\NormalTok{, .}\DecValTok{25}\NormalTok{, .}\DecValTok{25}\NormalTok{)}
\FunctionTok{fei}\NormalTok{(O, }\AttributeTok{p =}\NormalTok{ p\_E)}
\end{Highlighting}
\end{Shaded}

\begin{verbatim}
Fei  |       95% CI
-------------------
0.60 | [0.51, 1.00]

- Adjusted for non-uniform expected probabilities.
- One-sided CIs: upper bound fixed at [1.00].
\end{verbatim}

When there are only 2 cells with uniform expected probabilities (50\%),
this expression reduces to \(N\) and פ \(= w\).

\begin{Shaded}
\begin{Highlighting}[]
\NormalTok{O }\OtherTok{\textless{}{-}} \FunctionTok{c}\NormalTok{(}\DecValTok{90}\NormalTok{, }\DecValTok{10}\NormalTok{)}
\NormalTok{p\_E }\OtherTok{\textless{}{-}} \FunctionTok{c}\NormalTok{(}\FloatTok{0.5}\NormalTok{, }\FloatTok{0.5}\NormalTok{)}

\FunctionTok{fei}\NormalTok{(O, }\AttributeTok{p =}\NormalTok{ p\_E)}
\end{Highlighting}
\end{Shaded}

\begin{verbatim}
Fei  |       95% CI
-------------------
0.80 | [0.64, 1.00]

- One-sided CIs: upper bound fixed at [1.00].
\end{verbatim}

\begin{Shaded}
\begin{Highlighting}[]
\FunctionTok{cohens\_w}\NormalTok{(O, }\AttributeTok{p =}\NormalTok{ p\_E)}
\end{Highlighting}
\end{Shaded}

\begin{verbatim}
Cohen's w |       95% CI
------------------------
0.80      | [0.64, 1.00]

- One-sided CIs: upper bound fixed at [1.00].
\end{verbatim}

\hypertarget{summary}{%
\section{Summary}\label{summary}}

Effect sizes are essential to interpret the magnitude of observed
effects, they are frequently required in scientific journals, and they
are are necessary for a cumulative quantitative science relying on
meta-analyses. In this paper, we have covered the mathematics and
implementation in R of four different effect sizes for analyses of
categorical variables that specifically use the \(\chi^2\) (chi-square)
statistic. Furthermore, with our proposal of the effect size פ (Fei), we
fill the missing effect size for all cases of a \(\chi^2\) test - we now
have effect sizes that range from 0 to 1, that represent the sample's
\(\chi^2\) relative to the maximally possible \(\chi^2\) for contingency
tables that are 2-dimensional 2-by-2 (\(\phi\)) or larger (\(V\) or
\(T\)), and for 1-dimensional uniform 2-class (\emph{w}) or larger (פ).

% %%%%%%%%%%%%%%%%%%%%%%%%%%%%%%%%%%%%%%%%%%
% %% optional
% \supplementary{The following are available online at www.mdpi.com/link, Figure S1: title, Table S1: title, Video S1: title.}
%
% % Only for the journal Methods and Protocols:
% % If you wish to submit a video article, please do so with any other supplementary material.
% % \supplementary{The following are available at www.mdpi.com/link: Figure S1: title, Table S1: title, Video S1: title. A supporting video article is available at doi: link.}

\vspace{6pt}

%%%%%%%%%%%%%%%%%%%%%%%%%%%%%%%%%%%%%%%%%%
\acknowledgments{\emph{\{effectsize\}} is part of the collaborative
\href{https://github.com/easystats/easystats}{\emph{easystats}}
ecosystem \citep{easystatspackage}. Thus, we thank all
\href{https://github.com/orgs/easystats/people}{members of easystats},
contributors, and users alike.}

%%%%%%%%%%%%%%%%%%%%%%%%%%%%%%%%%%%%%%%%%%
\authorcontributions{ToDo}

%%%%%%%%%%%%%%%%%%%%%%%%%%%%%%%%%%%%%%%%%%
\conflictsofinterest{The authors declare no conflict of interest.}

%%%%%%%%%%%%%%%%%%%%%%%%%%%%%%%%%%%%%%%%%%
%% optional
\abbreviations{The following abbreviations are used in this manuscript:\\

\noindent
\begin{tabular}{@{}ll}
ToDo & ToDo \\
\end{tabular}}


%%%%%%%%%%%%%%%%%%%%%%%%%%%%%%%%%%%%%%%%%%
% Citations and References in Supplementary files are permitted provided that they also appear in the reference list here.

%=====================================
% References, variant A: internal bibliography
%=====================================
%\reftitle{References}
%\begin{thebibliography}{999}
% Reference 1
%\bibitem[Author1(year)]{ref-journal}
%Author1, T. The title of the cited article. {\em Journal Abbreviation} {\bf 2008}, {\em 10}, 142--149.
% Reference 2
%\bibitem[Author2(year)]{ref-book}
%Author2, L. The title of the cited contribution. In {\em The Book Title}; Editor1, F., Editor2, A., Eds.; Publishing House: City, Country, 2007; pp. 32--58.
%\end{thebibliography}

% The following MDPI journals use author-date citation: Arts, Econometrics, Economies, Genealogy, Humanities, IJFS, JRFM, Laws, Religions, Risks, Social Sciences. For those journals, please follow the formatting guidelines on http://www.mdpi.com/authors/references
% To cite two works by the same author: \citeauthor{ref-journal-1a} (\citeyear{ref-journal-1a}, \citeyear{ref-journal-1b}). This produces: Whittaker (1967, 1975)
% To cite two works by the same author with specific pages: \citeauthor{ref-journal-3a} (\citeyear{ref-journal-3a}, p. 328; \citeyear{ref-journal-3b}, p.475). This produces: Wong (1999, p. 328; 2000, p. 475)

%=====================================
% References, variant B: external bibliography
%=====================================
\reftitle{References}
\externalbibliography{yes}
\bibliography{paper.bib}

%%%%%%%%%%%%%%%%%%%%%%%%%%%%%%%%%%%%%%%%%%
%% optional

%% for journal Sci
%\reviewreports{\\
%Reviewer 1 comments and authors’ response\\
%Reviewer 2 comments and authors’ response\\
%Reviewer 3 comments and authors’ response
%}

%%%%%%%%%%%%%%%%%%%%%%%%%%%%%%%%%%%%%%%%%%


\end{document}
