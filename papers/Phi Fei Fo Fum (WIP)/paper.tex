% Options for packages loaded elsewhere
\PassOptionsToPackage{unicode}{hyperref}
\PassOptionsToPackage{hyphens}{url}
%
\documentclass[
]{article}
\usepackage{amsmath,amssymb}
\usepackage{lmodern}
\usepackage{iftex}
\ifPDFTeX
  \usepackage[T1]{fontenc}
  \usepackage[utf8]{inputenc}
  \usepackage{textcomp} % provide euro and other symbols
\else % if luatex or xetex
  \usepackage{unicode-math}
  \defaultfontfeatures{Scale=MatchLowercase}
  \defaultfontfeatures[\rmfamily]{Ligatures=TeX,Scale=1}
  \setmainfont[]{Times New Roman}
\fi
% Use upquote if available, for straight quotes in verbatim environments
\IfFileExists{upquote.sty}{\usepackage{upquote}}{}
\IfFileExists{microtype.sty}{% use microtype if available
  \usepackage[]{microtype}
  \UseMicrotypeSet[protrusion]{basicmath} % disable protrusion for tt fonts
}{}
\makeatletter
\@ifundefined{KOMAClassName}{% if non-KOMA class
  \IfFileExists{parskip.sty}{%
    \usepackage{parskip}
  }{% else
    \setlength{\parindent}{0pt}
    \setlength{\parskip}{6pt plus 2pt minus 1pt}}
}{% if KOMA class
  \KOMAoptions{parskip=half}}
\makeatother
\usepackage{xcolor}
\usepackage[margin=1in]{geometry}
\usepackage{color}
\usepackage{fancyvrb}
\newcommand{\VerbBar}{|}
\newcommand{\VERB}{\Verb[commandchars=\\\{\}]}
\DefineVerbatimEnvironment{Highlighting}{Verbatim}{commandchars=\\\{\}}
% Add ',fontsize=\small' for more characters per line
\usepackage{framed}
\definecolor{shadecolor}{RGB}{248,248,248}
\newenvironment{Shaded}{\begin{snugshade}}{\end{snugshade}}
\newcommand{\AlertTok}[1]{\textcolor[rgb]{0.94,0.16,0.16}{#1}}
\newcommand{\AnnotationTok}[1]{\textcolor[rgb]{0.56,0.35,0.01}{\textbf{\textit{#1}}}}
\newcommand{\AttributeTok}[1]{\textcolor[rgb]{0.77,0.63,0.00}{#1}}
\newcommand{\BaseNTok}[1]{\textcolor[rgb]{0.00,0.00,0.81}{#1}}
\newcommand{\BuiltInTok}[1]{#1}
\newcommand{\CharTok}[1]{\textcolor[rgb]{0.31,0.60,0.02}{#1}}
\newcommand{\CommentTok}[1]{\textcolor[rgb]{0.56,0.35,0.01}{\textit{#1}}}
\newcommand{\CommentVarTok}[1]{\textcolor[rgb]{0.56,0.35,0.01}{\textbf{\textit{#1}}}}
\newcommand{\ConstantTok}[1]{\textcolor[rgb]{0.00,0.00,0.00}{#1}}
\newcommand{\ControlFlowTok}[1]{\textcolor[rgb]{0.13,0.29,0.53}{\textbf{#1}}}
\newcommand{\DataTypeTok}[1]{\textcolor[rgb]{0.13,0.29,0.53}{#1}}
\newcommand{\DecValTok}[1]{\textcolor[rgb]{0.00,0.00,0.81}{#1}}
\newcommand{\DocumentationTok}[1]{\textcolor[rgb]{0.56,0.35,0.01}{\textbf{\textit{#1}}}}
\newcommand{\ErrorTok}[1]{\textcolor[rgb]{0.64,0.00,0.00}{\textbf{#1}}}
\newcommand{\ExtensionTok}[1]{#1}
\newcommand{\FloatTok}[1]{\textcolor[rgb]{0.00,0.00,0.81}{#1}}
\newcommand{\FunctionTok}[1]{\textcolor[rgb]{0.00,0.00,0.00}{#1}}
\newcommand{\ImportTok}[1]{#1}
\newcommand{\InformationTok}[1]{\textcolor[rgb]{0.56,0.35,0.01}{\textbf{\textit{#1}}}}
\newcommand{\KeywordTok}[1]{\textcolor[rgb]{0.13,0.29,0.53}{\textbf{#1}}}
\newcommand{\NormalTok}[1]{#1}
\newcommand{\OperatorTok}[1]{\textcolor[rgb]{0.81,0.36,0.00}{\textbf{#1}}}
\newcommand{\OtherTok}[1]{\textcolor[rgb]{0.56,0.35,0.01}{#1}}
\newcommand{\PreprocessorTok}[1]{\textcolor[rgb]{0.56,0.35,0.01}{\textit{#1}}}
\newcommand{\RegionMarkerTok}[1]{#1}
\newcommand{\SpecialCharTok}[1]{\textcolor[rgb]{0.00,0.00,0.00}{#1}}
\newcommand{\SpecialStringTok}[1]{\textcolor[rgb]{0.31,0.60,0.02}{#1}}
\newcommand{\StringTok}[1]{\textcolor[rgb]{0.31,0.60,0.02}{#1}}
\newcommand{\VariableTok}[1]{\textcolor[rgb]{0.00,0.00,0.00}{#1}}
\newcommand{\VerbatimStringTok}[1]{\textcolor[rgb]{0.31,0.60,0.02}{#1}}
\newcommand{\WarningTok}[1]{\textcolor[rgb]{0.56,0.35,0.01}{\textbf{\textit{#1}}}}
\usepackage{graphicx}
\makeatletter
\def\maxwidth{\ifdim\Gin@nat@width>\linewidth\linewidth\else\Gin@nat@width\fi}
\def\maxheight{\ifdim\Gin@nat@height>\textheight\textheight\else\Gin@nat@height\fi}
\makeatother
% Scale images if necessary, so that they will not overflow the page
% margins by default, and it is still possible to overwrite the defaults
% using explicit options in \includegraphics[width, height, ...]{}
\setkeys{Gin}{width=\maxwidth,height=\maxheight,keepaspectratio}
% Set default figure placement to htbp
\makeatletter
\def\fps@figure{htbp}
\makeatother
\setlength{\emergencystretch}{3em} % prevent overfull lines
\providecommand{\tightlist}{%
  \setlength{\itemsep}{0pt}\setlength{\parskip}{0pt}}
\setcounter{secnumdepth}{-\maxdimen} % remove section numbering
\newlength{\cslhangindent}
\setlength{\cslhangindent}{1.5em}
\newlength{\csllabelwidth}
\setlength{\csllabelwidth}{3em}
\newlength{\cslentryspacingunit} % times entry-spacing
\setlength{\cslentryspacingunit}{\parskip}
\newenvironment{CSLReferences}[2] % #1 hanging-ident, #2 entry spacing
 {% don't indent paragraphs
  \setlength{\parindent}{0pt}
  % turn on hanging indent if param 1 is 1
  \ifodd #1
  \let\oldpar\par
  \def\par{\hangindent=\cslhangindent\oldpar}
  \fi
  % set entry spacing
  \setlength{\parskip}{#2\cslentryspacingunit}
 }%
 {}
\usepackage{calc}
\newcommand{\CSLBlock}[1]{#1\hfill\break}
\newcommand{\CSLLeftMargin}[1]{\parbox[t]{\csllabelwidth}{#1}}
\newcommand{\CSLRightInline}[1]{\parbox[t]{\linewidth - \csllabelwidth}{#1}\break}
\newcommand{\CSLIndent}[1]{\hspace{\cslhangindent}#1}
\ifLuaTeX
  \usepackage{selnolig}  % disable illegal ligatures
\fi
\IfFileExists{bookmark.sty}{\usepackage{bookmark}}{\usepackage{hyperref}}
\IfFileExists{xurl.sty}{\usepackage{xurl}}{} % add URL line breaks if available
\urlstyle{same} % disable monospaced font for URLs
\hypersetup{
  pdftitle={Phi, Fei, Fo, Fum: Effect Sizes for Chi-squared Tests},
  hidelinks,
  pdfcreator={LaTeX via pandoc}}

\title{Phi, Fei, Fo, Fum: Effect Sizes for Chi-squared Tests}
\author{}
\date{\vspace{-2.5em}}

\begin{document}
\maketitle

\hypertarget{introduction}{%
\section{Introduction}\label{introduction}}

In both theoretical and applied research, it is often of interest to
assess the strength of an observed association. Beyond their use in
interpreting the results from a given study, they can be aggregated in
meta-analyses (Lakens, 2013). We review here the most common effect
sizes for analyses of categorical variables that use the \(\chi^2\)
(chi-square) statistic, and introduce a new one---פ (Fei)---alongside
the \texttt{\{effectsize\}} package (Ben-Shachar, Lüdecke, \& Makowski,
2020) in the R programming language (R Core Team, 2023), which
implements these measures and their confidence intervals.

\hypertarget{tests-of-independence}{%
\section{Tests of Independence}\label{tests-of-independence}}

The \(\chi^2\) test of independence between two categorical variables
examines if the frequency distribution of one of the variables is
dependent on the other. Formally, the test examines how likely the
observed conditional frequencies (cell frequencies) are under the null
hypotheses of independence. This is done by examining the degree the
observed cell frequencies deviate from the frequencies that would be
expected if the variables were indeed independent. The test statistic
for these tests is the \(\chi^2\), which is computed as:

\[
\chi^2 = \sum_{i=1}^{l\times k}{\frac{(O_i-E_i)^2}{E_i}}
\]

Where \(O_i\) are the \emph{observed} frequencies and \(E_i\) are the
frequencies \emph{expected} under independence, and \(l\) and \(k\) are
the number of rows and columns of the contingency table.

Instead of the deviations between the observed and expected frequencies,
we can write \(\chi^2\) in terms of observed and expected cell
\emph{probabilities} and the total sample size \(N\) (since \(p=k/N\)):

\[
\chi^2 = N\times\sum_{i=1}^{l\times k}{\frac{(p_{O_i}-p_{E_i})^2}{p_{E_i}}}
\]

Where \(p_{O_i}\) are the \emph{observed} cell probabilities and
\(p_{E_i}\) are the probabilities \emph{expected} under independence.

For example, which might ask of the probability of survival of the
sinking of the Titanic is dependant on the sex of the passenger:

\begin{Shaded}
\begin{Highlighting}[]
\NormalTok{(Titanic\_xtab }\OtherTok{\textless{}{-}} \FunctionTok{as.table}\NormalTok{(}\FunctionTok{apply}\NormalTok{(Titanic, }\FunctionTok{c}\NormalTok{(}\DecValTok{2}\NormalTok{, }\DecValTok{4}\NormalTok{), sum)))}
\end{Highlighting}
\end{Shaded}

\begin{verbatim}
##         Survived
## Sex        No  Yes
##   Male   1364  367
##   Female  126  344
\end{verbatim}

\begin{Shaded}
\begin{Highlighting}[]
\FunctionTok{chisq.test}\NormalTok{(Titanic\_xtab)}
\end{Highlighting}
\end{Shaded}

\begin{verbatim}
## 
##  Pearson's Chi-squared test with Yates' continuity correction
## 
## data:  Titanic_xtab
## X-squared = 454.5, df = 1, p-value < 0.00000000000000022
\end{verbatim}

\hypertarget{phi}{%
\subsection{Phi}\label{phi}}

For a 2-by-2 contingency table analysis, like the one used above, the
\(\phi\) (\emph{phi}) coefficient is a correlation-like measure of
effect size indicating the strength of association between the two
binary variables. One way to compute this effect size is to re-code the
binary variables as dummy (0, 1) variables, and computing the (absolute)
Pearson correlation between them:

\[
\phi = |r_{AB}|
\]

Another way to compute \(\phi\) is by using the \(\chi^2\) statistic:

\[
\phi = \sqrt{\frac{\chi^2}{N}} = \sqrt{\sum_{i=1}^{l\times k}{\frac{(p_{O_i}-p_{E_i})^2}{p_{E_i}}}}
\]

This value ranges between 0 (no association) and 1 (complete
dependence), and its values can be interpreted the same as Person's
correlation coefficient.

\begin{Shaded}
\begin{Highlighting}[]
\FunctionTok{library}\NormalTok{(effectsize)}
\FunctionTok{library}\NormalTok{(correlation)}

\FunctionTok{phi}\NormalTok{(Titanic\_xtab, }\AttributeTok{adjust =} \ConstantTok{FALSE}\NormalTok{)}
\end{Highlighting}
\end{Shaded}

\begin{verbatim}
## Phi  |       95% CI
## -------------------
## 0.46 | [0.42, 1.00]
## 
## - One-sided CIs: upper bound fixed at [1.00].
\end{verbatim}

\begin{Shaded}
\begin{Highlighting}[]
\NormalTok{tidyr}\SpecialCharTok{::}\FunctionTok{uncount}\NormalTok{(}\FunctionTok{as.data.frame}\NormalTok{(Titanic\_xtab), }\AttributeTok{weights =}\NormalTok{ Freq) }\SpecialCharTok{|\textgreater{}}
  \FunctionTok{transform}\NormalTok{(}\AttributeTok{Survived =}\NormalTok{ Survived }\SpecialCharTok{==} \StringTok{"Yes"}\NormalTok{,}
            \AttributeTok{Sex =}\NormalTok{ Sex }\SpecialCharTok{==} \StringTok{"Male"}\NormalTok{) }\SpecialCharTok{|\textgreater{}} 
  \FunctionTok{correlation}\NormalTok{()}
\end{Highlighting}
\end{Shaded}

\begin{verbatim}
## # Correlation Matrix (pearson-method)
## 
## Parameter1 | Parameter2 |     r |         95% CI | t(2199) |         p
## ----------------------------------------------------------------------
## Sex        |   Survived | -0.46 | [-0.49, -0.42] |  -24.00 | < .001***
## 
## p-value adjustment method: Holm (1979)
## Observations: 2201
\end{verbatim}

\hypertarget{cramuxe9rs-v}{%
\subsection{\texorpdfstring{Cramér's
\emph{V}}{Cramér's V}}\label{cramuxe9rs-v}}

These properties do not hold when the contingency table is larger than
2-by-2: \(\sqrt{\chi^2/N}\) can be larger than 1. Cramér showed (Cramér,
1999) that while for 2-by-2 the maximal possible value of \(\chi^2\) is
\(N\), for larger tables the maximal possible value for \(\chi^2\) is
\(N\times (\text{min}(k,l)-1)\). Therefore, he suggested the \(V\)
effect size (also sometimes known as Cramér's phi and denoted as
\(\phi_{c}\)):

\[
\text{Cramer's } V = \sqrt{\frac{\chi^2}{N(\text{min}(k,l)-1)}}
\]

\(V\) is 1 when the columns are completely dependent on the rows, or the
row are completely dependent on the columns.

\begin{Shaded}
\begin{Highlighting}[]
\NormalTok{(Titanic\_xtab2 }\OtherTok{\textless{}{-}} \FunctionTok{as.table}\NormalTok{(}\FunctionTok{apply}\NormalTok{(Titanic, }\FunctionTok{c}\NormalTok{(}\DecValTok{1}\NormalTok{, }\DecValTok{4}\NormalTok{), sum)))}
\end{Highlighting}
\end{Shaded}

\begin{verbatim}
##       Survived
## Class   No Yes
##   1st  122 203
##   2nd  167 118
##   3rd  528 178
##   Crew 673 212
\end{verbatim}

\begin{Shaded}
\begin{Highlighting}[]
\FunctionTok{cramers\_v}\NormalTok{(Titanic\_xtab2, }\AttributeTok{adjust =} \ConstantTok{FALSE}\NormalTok{)}
\end{Highlighting}
\end{Shaded}

\begin{verbatim}
## Cramer's V |       95% CI
## -------------------------
## 0.29       | [0.26, 1.00]
## 
## - One-sided CIs: upper bound fixed at [1.00].
\end{verbatim}

Tschuprow (Tschuprow, 1939) devised an alternative value, at , which is
1 only when the columns are completely dependent on the rows \emph{and}
the rows are completely dependent on the columns, which is only possible
when the contingency table is a square.

\begin{Shaded}
\begin{Highlighting}[]
\FunctionTok{tschuprows\_t}\NormalTok{(Titanic\_xtab2)}
\end{Highlighting}
\end{Shaded}

\begin{verbatim}
## Tschuprow's T |       95% CI
## ----------------------------
## 0.22          | [0.20, 1.00]
## 
## - One-sided CIs: upper bound fixed at [1.00].
\end{verbatim}

We can generalize both \(\phi\), \(V\), and \(T\) to:
\(\sqrt{\frac{\chi^2}{\chi^2_{\text{max}}}}\).

\hypertarget{goodness-of-fit}{%
\section{Goodness of Fit}\label{goodness-of-fit}}

These tests compare an observed distribution of a multinomial variable
to an expected distribution, using the same \(\chi^2\) statistic. Here
too we can compute an effect size as
\(\sqrt{\frac{\chi^2}{\chi^2_{\text{max}}}}\), all we need to find is
\(\chi^2_{\text{max}}\).

\hypertarget{cohens-w}{%
\subsection{\texorpdfstring{Cohen's
\emph{w}}{Cohen's w}}\label{cohens-w}}

Cohen (Cohen, 2013) defined an effect size---\emph{w}---for the goodness
of fit test:

\[
\text{Cohen's } w = \sqrt{\sum_{i=1}^{k}{\frac{(p_{O_i}-p_{E_i})^2}{p_{E_i}}}} = \sqrt{\frac{\chi^2}{N}}
\]

Thus, \(\chi^2_\text{max} = N\).

\begin{Shaded}
\begin{Highlighting}[]
\NormalTok{(Titanic\_freq }\OtherTok{\textless{}{-}} \FunctionTok{as.table}\NormalTok{(}\FunctionTok{apply}\NormalTok{(Titanic, }\DecValTok{2}\NormalTok{, sum)))}
\end{Highlighting}
\end{Shaded}

\begin{verbatim}
##   Male Female 
##   1731    470
\end{verbatim}

\begin{Shaded}
\begin{Highlighting}[]
\NormalTok{p\_E }\OtherTok{\textless{}{-}} \FunctionTok{c}\NormalTok{(}\FloatTok{0.5}\NormalTok{, }\FloatTok{0.5}\NormalTok{)}

\FunctionTok{cohens\_w}\NormalTok{(Titanic\_freq, }\AttributeTok{p =}\NormalTok{ p\_E)}
\end{Highlighting}
\end{Shaded}

\begin{verbatim}
## Cohen's w |       95% CI
## ------------------------
## 0.57      | [0.54, 1.00]
## 
## - One-sided CIs: upper bound fixed at [1.00].
\end{verbatim}

Unfortunately, \emph{w} has an upper bound of 1 \emph{only} when the
variable is binomial (has two categories) and the expected distribution
is uniform (\(p = 1 - p = 0.5\)). If the distribution is none uniform
(Rosenberg, 2010) or if there are more than 2 classes (Johnston, Berry,
\& Mielke Jr, 2006), then \(\chi^2_\text{max} > N\), and so \emph{w} can
be larger than 1.

\begin{Shaded}
\begin{Highlighting}[]
\NormalTok{O }\OtherTok{\textless{}{-}} \FunctionTok{c}\NormalTok{(}\DecValTok{90}\NormalTok{, }\DecValTok{10}\NormalTok{)}
\NormalTok{p\_E }\OtherTok{\textless{}{-}} \FunctionTok{c}\NormalTok{(}\FloatTok{0.35}\NormalTok{, }\FloatTok{0.65}\NormalTok{)}
\FunctionTok{cohens\_w}\NormalTok{(O, }\AttributeTok{p =}\NormalTok{ p\_E)}
\end{Highlighting}
\end{Shaded}

\begin{verbatim}
## Cohen's w |       95% CI
## ------------------------
## 1.15      | [0.99, 1.36]
## 
## - One-sided CIs: upper bound fixed at [1.36~].
\end{verbatim}

\begin{Shaded}
\begin{Highlighting}[]
\NormalTok{O }\OtherTok{\textless{}{-}} \FunctionTok{c}\NormalTok{(}\DecValTok{10}\NormalTok{, }\DecValTok{20}\NormalTok{, }\DecValTok{80}\NormalTok{, }\DecValTok{5}\NormalTok{)}
\NormalTok{p\_E }\OtherTok{\textless{}{-}} \FunctionTok{c}\NormalTok{(.}\DecValTok{25}\NormalTok{, .}\DecValTok{25}\NormalTok{, .}\DecValTok{25}\NormalTok{, .}\DecValTok{25}\NormalTok{)}
\FunctionTok{cohens\_w}\NormalTok{(O, }\AttributeTok{p =}\NormalTok{ p\_E)}
\end{Highlighting}
\end{Shaded}

\begin{verbatim}
## Cohen's w |       95% CI
## ------------------------
## 1.05      | [0.88, 1.73]
## 
## - One-sided CIs: upper bound fixed at [1.73~].
\end{verbatim}

\hypertarget{fei}{%
\subsection{Fei}\label{fei}}

We present here a new effect size, פ (Fei), which normalizes
goodness-of-fit \(\chi^2\) by the proper \(\chi^2_\text{max}\) for
non-uniform and/or multinomial variables.

The largest deviation from the expected probability distribution would
occur when all observations are in the cell with the smallest expected
probability. That is:

\[
p_{O} = 
\begin{cases}
1, & \text{if } p_i = \text{min}(p) \\
0, & \text{Otherwise}
\end{cases}
\]

Since
\(\chi^2 = N \times\sum_{i=1}^{k}{\frac{(p_{E_i}-p_{O_i})^2}{p_{E_i}}}\),
we can find \(\frac{(E_i-O_i)^2}{E_i}\) for each of these values:

\[
\frac{(p_{E}-p_{O})^2}{p_{E}} = 
\begin{cases}
\frac{(p_i-1)^2}{p_i}, & \text{if } p_{E} = \text{min}(p_{E}) \\
p_i = \frac{(p_i-0)^2}{p_i}, & \text{Otherwise}
\end{cases}
\]

Since \(\sum_{i=1}^{k}{p_i}=1\), the \(\chi^2\), which is the sum of the
expression above, can be retrieved as:

\[
\begin{split}
\chi^2_\text{max} & = N \times (1 - \text{min}(p_E) + \frac{(\text{min}(p_E)-1)^2}{\text{min}(p_E)}) \\
 & = N \times \frac{1-\text{min}(p_E)}{\text{min}(p_E)} \\
 & = N \times (\frac{1}{\text{min}(p_E)} - 1)
\end{split}
\]

And so an effect size can be derived as:

\[
\sqrt{\frac{\chi^2}{N \times (\frac{1}{\text{min}(p_E)} - 1)}}
\]

We call this effect size פ (Fei), which represents the voiceless
bilabial fricative in the Hebrew language, keeping in line with \(\phi\)
(which in modern Greek marks the same sound) and \(V\) (which in English
marks a voiced bilabial fricative; \(W\) being derived from the letter V
in modern Latin alphabet). פ will be 0 when the observed distribution
matches the expected one perfectly, and will be 1 when the observed
values are all of the same class---the one with the smallest expected
probability.

\begin{Shaded}
\begin{Highlighting}[]
\NormalTok{O }\OtherTok{\textless{}{-}} \FunctionTok{c}\NormalTok{(}\DecValTok{90}\NormalTok{, }\DecValTok{10}\NormalTok{)}
\NormalTok{p\_E }\OtherTok{\textless{}{-}} \FunctionTok{c}\NormalTok{(}\FloatTok{0.35}\NormalTok{, }\FloatTok{0.65}\NormalTok{)}
\FunctionTok{fei}\NormalTok{(O, }\AttributeTok{p =}\NormalTok{ p\_E)}
\end{Highlighting}
\end{Shaded}

\begin{verbatim}
## Fei  |       95% CI
## -------------------
## 0.85 | [0.73, 1.00]
## 
## - Adjusted for uniform expected probabilities.
## - One-sided CIs: upper bound fixed at [1.00].
\end{verbatim}

\begin{Shaded}
\begin{Highlighting}[]
\NormalTok{O }\OtherTok{\textless{}{-}} \FunctionTok{c}\NormalTok{(}\DecValTok{10}\NormalTok{, }\DecValTok{20}\NormalTok{, }\DecValTok{80}\NormalTok{, }\DecValTok{5}\NormalTok{)}
\NormalTok{p\_E }\OtherTok{\textless{}{-}} \FunctionTok{c}\NormalTok{(.}\DecValTok{25}\NormalTok{, .}\DecValTok{25}\NormalTok{, .}\DecValTok{25}\NormalTok{, .}\DecValTok{25}\NormalTok{)}
\FunctionTok{fei}\NormalTok{(O, }\AttributeTok{p =}\NormalTok{ p\_E)}
\end{Highlighting}
\end{Shaded}

\begin{verbatim}
## Fei  |       95% CI
## -------------------
## 0.60 | [0.51, 1.00]
## 
## - Adjusted for non-uniform expected probabilities.
## - One-sided CIs: upper bound fixed at [1.00].
\end{verbatim}

When there are only 2 cells with uniform expected probabilities (50\%),
this expression reduces to \(N\) and פ \(= w\).

\begin{Shaded}
\begin{Highlighting}[]
\NormalTok{O }\OtherTok{\textless{}{-}} \FunctionTok{c}\NormalTok{(}\DecValTok{90}\NormalTok{, }\DecValTok{10}\NormalTok{)}
\NormalTok{p\_E }\OtherTok{\textless{}{-}} \FunctionTok{c}\NormalTok{(}\FloatTok{0.5}\NormalTok{, }\FloatTok{0.5}\NormalTok{)}

\FunctionTok{fei}\NormalTok{(O, }\AttributeTok{p =}\NormalTok{ p\_E)}
\end{Highlighting}
\end{Shaded}

\begin{verbatim}
## Fei  |       95% CI
## -------------------
## 0.80 | [0.64, 1.00]
## 
## - One-sided CIs: upper bound fixed at [1.00].
\end{verbatim}

\begin{Shaded}
\begin{Highlighting}[]
\FunctionTok{cohens\_w}\NormalTok{(O, }\AttributeTok{p =}\NormalTok{ p\_E)}
\end{Highlighting}
\end{Shaded}

\begin{verbatim}
## Cohen's w |       95% CI
## ------------------------
## 0.80      | [0.64, 1.00]
## 
## - One-sided CIs: upper bound fixed at [1.00].
\end{verbatim}

\hypertarget{summary}{%
\section{Summary}\label{summary}}

We fill the missing effect size for all cases of a \(\chi^2\) test.

\hypertarget{references}{%
\section*{References}\label{references}}
\addcontentsline{toc}{section}{References}

\hypertarget{refs}{}
\begin{CSLReferences}{1}{0}
\leavevmode\vadjust pre{\hypertarget{ref-benshachar2020effectsize}{}}%
Ben-Shachar, M. S., Lüdecke, D., \& Makowski, D. (2020). {e}ffectsize:
Estimation of effect size indices and standardized parameters.
\emph{Journal of Open Source Software}, \emph{5}(56), 2815.
\url{https://doi.org/10.21105/joss.02815}

\leavevmode\vadjust pre{\hypertarget{ref-cohen2013statistical}{}}%
Cohen, J. (2013). \emph{Statistical power analysis for the behavioral
sciences}. Routledge.

\leavevmode\vadjust pre{\hypertarget{ref-cramer1999mathematical}{}}%
Cramér, H. (1999). \emph{Mathematical methods of statistics} (Vol. 43).
Princeton University Press.

\leavevmode\vadjust pre{\hypertarget{ref-johnston2006measures}{}}%
Johnston, J. E., Berry, K. J., \& Mielke Jr, P. W. (2006). Measures of
effect size for chi-squared and likelihood-ratio goodness-of-fit tests.
\emph{Perceptual and Motor Skills}, \emph{103}(2), 412--414.

\leavevmode\vadjust pre{\hypertarget{ref-lakens2013calculating}{}}%
Lakens, D. (2013). Calculating and reporting effect sizes to facilitate
cumulative science: A practical primer for \emph{t}-tests and ANOVAs.
\emph{Frontiers in Psychology}, \emph{4}.
\url{https://doi.org/10.3389/fpsyg.2013.00863}

\leavevmode\vadjust pre{\hypertarget{ref-base2023}{}}%
R Core Team. (2023). \emph{R: A language and environment for statistical
computing}. Retrieved from \url{https://www.R-project.org/}

\leavevmode\vadjust pre{\hypertarget{ref-rosenberg2010generalized}{}}%
Rosenberg, M. S. (2010). A generalized formula for converting chi-square
tests to effect sizes for meta-analysis. \emph{PloS One}, \emph{5}(4),
e10059.

\leavevmode\vadjust pre{\hypertarget{ref-tschuprow1939principles}{}}%
Tschuprow, A. A. (1939). \emph{Principles of the mathematical theory of
correlation}. Hodge.

\end{CSLReferences}

\end{document}
